\documentclass[leqno, 12pt, twoside,letterpaper]{book}
\usepackage{makeidx}

\usepackage[T1]{fontenc}
\usepackage[utf8]{inputenc}%
\usepackage{amsthm}
\usepackage{titlesec}
\usepackage{amsmath}
\usepackage{charter}
\usepackage[bitstream-charter]{mathdesign}
\usepackage{framed}
%\usepackage[a4paper]{geometry}
\usepackage[right=1.1in, left=1.2in]{geometry}
\usepackage{bm}
\usepackage{enumitem}
\usepackage[strict]{changepage}
\usepackage[most]{tcolorbox}
\usepackage{multicol}


\theoremstyle{plain}
\newtheorem{thm}{Teorema}[chapter]
\newtheorem{lem}[thm]{Lema}
\newtheorem{cor}[thm]{Corolario}

\theoremstyle{definition}
\newtheorem*{ex}{Ejemplo}
\newtheorem*{exs}{Ejemplos}
\newtheorem*{thmx}{Teorema}
\newtheorem{propi}[thm]{Propiedad}
\newtheorem{prop}[thm]{Proposición}
%\newtheorem{ej}[thm]{Ejercicio}

\newtheorem*{notacion}{Notación:}

\newcommand{\va}[1][\mathbb]{#1}
\newcommand{\prob}[1][\mathbb]{#1}
\newcommand{\dist}[1][\mathcal]{#1}
\newcommand{\cond}[2]{\{ #1 \, | \, #2 \}}
\newcommand{\set}[2]{\{ #1  \mid  #2 \}}
\newcommand{\pset}[1]{\{ #1 \}}
\newcommand{\bgamma}[0]{\bm{\Gamma}}
\newcommand{\ba}[0]{\bm{A}}
\newcommand{\bb}[0]{\bm{B}}
\newcommand{\FORM}[0]{\bm{FORM}}
\newcommand{\naturals}[0]{\mathbb{N}}
\newcommand{\sii}[0]{\Longleftrightarrow}
\newcommand{\union}[0]{\cup}
\newcommand{\class}[1][\mathcal]{#1}
\newcommand{\PRC}[0]{\bm{PRC}}
\newcommand{\HALT}[2]{\bm{HALT}(#1, #2)}
\newcommand{\func}[3]{#1 \colon #2 \to #3}
\newcommand{\TeoN}[0]{\mbox{Teo}(\mathcal{N})}
\newcommand{\ARITH}[0]{\mathcal{N}}
\newcommand{\MODEL}[0]{\mathscr{M}}
\newcommand{\TOT}[0]{\bm{TOT}}
\newcommand{\STP}[0]{\mbox{STP}}
\newcommand{\NTOT}[0]{\bm{\overline{TOT}}}
\newcommand{\dom}[0]{\mbox{dom}}
\newcommand{\piz}[0]{\psi \rightarrow \zeta}
\newcommand{\fis}[0]{\varphi \rightarrow \psi}
\newcounter{excounter}

\newenvironment{solucion}[0]{\begin{adjustwidth}{1.5em}{1.5em}}{\end{adjustwidth}}
\newenvironment{teo}[1]{\refstepcounter{thm}\noindent\textbf{Teorema~\thethm  #1:} }{}
\newenvironment{dem}[0]{\noindent\textbf{Demostración:}}{}
\newenvironment{defi}[0]{\refstepcounter{thm}\noindent\textbf{Definición:}}{}

\newtcolorbox{exbox}[0]{
                colback=white!80!gray,
				colframe=white}
			
\newenvironment{ej}[1]{\noindent\stepcounter{excounter} \begin{exbox}\textbf{\arabic{excounter}} (#1).}{\end{exbox}}
	
%\newtheorem{ej}{}

\newcommand{\twopartdef}[4]
{
	\left\{
		\begin{array}{ll}
			#1 & \mbox{si } #2 \\
			#3 & \mbox{} #4
		\end{array}
	\right.
}

\newcommand{\grab}[3]
{
	\left.
		\begin{array}{ll}
			#1 & \\
			#2
		\end{array}
	\right\} \implies #3
}

\newcommand{\twopartdeftwo}[4]
{
	\left\{
		\begin{array}{ll}
			#1 & \mbox{si } #2 \\
			\\
			#3 & \mbox{si } #4
		\end{array}
	\right.
}

\newenvironment{obs}
  {\begin{framed}}
  {\end{framed}}



\titleformat{\chapter}[display]{\normalfont\bfseries}{}{0pt}{\Huge}
\frenchspacing


\title{Ejercicios para el final de Lógica y Computabilidad.}

\begin{document}

\maketitle
\section*{Computabilidad}

\begin{ej}{13}
    Resolver los siguientes puntos:
    \begin{enumerate}
        \item Definir un conjunto de índices.
        \item Dar un ejemplo de un conjunto de índices y uno que no lo sea
        \item Enunciar y demostrar el teorema de Rice
    \end{enumerate}
\end{ej}

\begin{solucion}
\textbf{1.} Sea $\class{C}$ una clase de funciónes parciales computables. Entonces el conjunto de índices $A$ se define como
\[
	A = \set{x}{\phi_{x} \in \class{C}}
\]
Notar que si $x \neq y$, $x \in A\,$ y $\,\phi_{x} = \phi_{y}$ entonces $y \in A$.

\noindent\textbf{2.}

\begin{teo}{ (Rice)}Sea $A$ un conjunto de índices, entonces si $A \neq \emptyset$ ó $A \neq \naturals$ entonces $A$ no es computable. Es decir si $A$ no es un conjunto trivial entonces no es computable.
\end{teo}

\begin{dem}
	Supongamos que $A$ es computable, esto significa que tenemos una función $h$ computable que computa la pertenencia en $A$.

	\[
	h(x) = \twopartdef{1}{x \in A.}
					  {0}{\mbox{si no.}}
	\]

	\noindent Además como $A$ no es trivial por lo menos existen dos funciones $f \in \class{C}$ y $g \not\in \class{C}$ computables.

	\noindent Entonces definimos $d$ como

	\[
		d(t, x) = \twopartdef{g(x)}{h(t).}
							 {f(x)}{\mbox{si no.}}
	\]

	\noindent Esta función también es computable y por el teorema de la recursión existe $e \in \naturals$ tal que
	\[
	\phi_{e}(x) = d(e, x)
	\]

	\noindent Pero entonces tenemos que:

	\[ e \in A \implies \phi_{e}(x) = g(x) \implies \phi_{e}(x) \not\in \class{C} \implies e \not\in A\]
	\[ e \not\in A \implies \phi_{e}(x) = f(x) \implies \phi_{e}(x) \in \class{C} \implies e \in A\]

	\noindent Llegamos a una contradicción que vino de suponer que $A$ era computable.

\end{dem}
\end{solucion}

\begin{ej}{5}
    Demostrar que una función es $p.r.$ si y solo si pertenece a toda clase
    $\PRC$
\end{ej}

\begin{solucion}

\begin{defi} Una función es primitiva recursiva (p.r.) si se puede obtener a partir de las funciones iniciales por un número finito de aplicaciones de composición y recursión primitiva.
\end{defi}

\begin{dem}
$\Rightarrow)$ Sea $\class{C}$ una clase $\PRC$, como $f$ p.r. hay una lista

	\[ f_1, f_2, \cdots, f_n \]

	TODO(Inducción)

\noindent $\Leftarrow)$ Las funciones p.r. son una clase $\PRC$ en particular si $f$ pertenece a toda clase $\PRC$ tiene que pertenecer a la clase de las funciones p.r., osea que $f$ es p.r.
\end{dem}
\end{solucion}

\begin{ej}{4}
    Probar que $\bm{TOT}$ no es $c.e.$ ni $co-c.e.$
\end{ej}

\begin{solucion}

$\TOT$ no c.e.: Supongamos que $\TOT$ es c.e., entonces existe una función parcial computable total $f$ cuyo rango es $\TOT$. Existe $e$ tal que $\phi_{e}(x) = \phi_{f(x)}(x) + 1$. Pero entonces $e$ es total por que existe un $k \in \naturals$ tal que $f(k) = e$, entonces
\[ \phi_{e}(x) = \phi_{f(k)}(x) = \phi_{f(x)}(x) + 1 \] 

\noindent Si tomamos $x = k$ llegamos a una contradicción, vino de suponer que $\TOT$ es c.e. \\

\noindent $\NTOT$ no es c.e.: \[\NTOT = \set{x}{\phi_x \mbox{no es total}}\]
Supongamos que $\NTOT$ es c.e., existe un $e$ tal que $\mbox{dom}(\phi_{e}) = \NTOT$.
Definimos el siguiente programa $P$
\begin{align*}
	[C] \quad & \mbox{IF STP}(x, e, t) = 1 \mbox{ GOTO E} \\
		\quad & T \leftarrow T + 1 \\
		\quad & \mbox{GOTO C}
\end{align*}
\noindent Como $\NTOT = \mbox{dom}(\phi_e)$, este programa se indefine solo si $x$ es el número de un programa total computable. En otras palabras $P$ es equivalente a
\[ \Psi_{p}(x, y) = g(x, y) = \twopartdef{\uparrow}{\phi_x \mbox{ es total.}}
						 				 {0}{\mbox{si no.}}\]
\noindent Usando el teorema de la recursión hay un $e$ tal que $\phi_e(y) = g(e, y)$, pero entonces tenemos que para todo y,
 \[ e \in \TOT \implies  \phi_e(y) \uparrow \implies e \not\in \TOT \]  
 \[ e \not\in \TOT \implies \phi_e(y) = 0 \implies e \in\TOT \]
\end{solucion}

\begin{ej}{3}
    Enunciar el halting problem y demostrar que $\HALT{x}{y}$ no es computable.
\end{ej}

\begin{solucion}
\noindent $\func{\HALT{x}{y}}{\naturals^2}{\{0, 1\}} $ es la funcion que devuelve verdadero si el programa $x$ con entrada $y$ no se indefine:

	\[ \HALT{x}{y} = \twopartdef{1}{\psi_{P}(x) \downarrow}{0}{\mbox{si no.}}\]

\noindent El halting problem resulta de saber si esta función es o no computable. \\

\begin{teo}{} $\HALT{x}{y}$ no es computable.
\end{teo}

\begin{dem}
	Supongamos que lo es, entonces definimos la siguiente funcion $c$

	\[ c(x) = \twopartdef{\phi_x{(x)} + 1}{\HALT{x}{x}}{0}{\mbox{si no.}} \]

	\noindent Como $\HALT{x}{y}$ es computable entonces $c$ es computable y total, osea que $c(x) \downarrow$ para cualquier input. Además por universalidad existe un $e$ tal que $c(x) = \phi_{e}(x)$, tenemos que
	\[ \phi_{e}(x) = c(x) = \phi_{x}(x) + 1 \]

	\noindent pues $c$ es total. Si tomamos $x = e$ llegamos a una contradicción, vino de suponer que $\HALT{x}{y}$ es computable.
\end{dem}
\end{solucion}

\begin{ej}{2}
    $\ba \subseteq \naturals$ es computable $\sii$ $\ba$ y $\bar{\ba}$ son
    computablemente enumerables.
\end{ej}

\begin{solucion}

\begin{dem}
$\Rightarrow)$	Si $\ba$ es computable existe $c$ computable (?`Por qué?) tal que
		\[ c(x) = \twopartdef{1}{x \in \ba}{0}{\mbox{si no.}} \]

		\noindent Podemos además suponer que $\ba$ no es trivial ($\emptyset$ ó $\naturals$), entonces tomamos un elemento $k \in \ba$ y definimos
		\[ f(x) = \twopartdef{x}{c(x)}{k}{\mbox{si no.}} \]

	\noindent Como $c$ era computable $f$ es computable y además $f(\naturals) = \ba$. De la misma manera podemos construir $g$ tal que $g(\naturals) = \bar{\ba}$\\

	\noindent $\Leftarrow)$ Si $\ba$ y $\bar{\ba}$ son c.e. significa que tenemos $f$ y $g$ funciones computables tal que $f(\naturals) = \ba$ y $g(\naturals) = \bar{\ba}$, queremos una función que nos devuelva verdadero si un elemento pertenece a $\ba$, pero entonces es cuestion de probar si hay un $k \in \naturals$ tal que este en la imagen de $f$, no podemos loopear indefinidamente a riesgo de indefinirnos, por lo que hay que probar un elemento a la vez en $f$ y $g$, podemos definir entonces la siguiente función:
	\[ c(x) = \twopartdef{1}{f(\min\limits_{y}(f(y) = x \lor g(y) = x) = x}{0}{\mbox{si no.}} \]
\end{dem}

\noindent Esta función, a pesar de tener una minimización no acotada es total y devuelve true si la aparición de $x$ esta en $\ba$.
\end{solucion}

\begin{ej}{2}
    Enunciar y demostrar el teorema de la recursión (Hint: Teorema del parámetro).
\end{ej}

\begin{solucion}
\begin{teo}{}
	Si $\func{g}{\naturals^{n + 1}}{\naturals}$ es parcial computable, existe un $e$ tal que
	\[ \phi_e(x_1, \dots, x_n) = g(e, x_1, \dots, x_n)\]
\end{teo}

\begin{dem}
	Sea $S_n^1$ la función del teorema del parametro
	\[ \phi_{S_n^1(e, y)}(x_1, \dots, x_n) = \phi_r(x_1, \dots, x_n, e) \]
La función \[(x_1, \dots, x_n, v) \mapsto g(S_n^1(v, v), x_1, \dots, x_n)\] 
es parcial computable, de modo que existe $c$ tal que
\begin{align*}
	g(S_n^1(v, v), x_1, \dots, x_n) &= \phi_c(x_1, \dots, x_n, v)\\
									&= \phi_{S_n^1(v, c)}(x_1, \dots, x_n) 
\end{align*}
$c$ está fijo, pero $v$ es variable, si elegimos $v = c$ y $e = S_n^1(c, c)$ llegamos a lo que queríamos.

\end{dem}
\end{solucion}

\begin{ej}{2}
    Demostrar que si $\bm{A}$ es $c.e.$ y $\bm{A} \neq \emptyset$, entonces
    $\bm{A}$ es el rango de una función p.r.
\end{ej}

\begin{solucion}
	Como $A$ es c.e. por definición $A = \set{x}{ g(x) \downarrow}$ donde $\func{g}{\naturals}{\naturals}$ es una función parcial computable, además como $A$ no es vacio por lo menos tiene un elemento $k$. Sea $e$ tal que $\phi_{e} = g$, entonces podemos definir la siguiente funcion, 
	\[ f(\langle x, t \rangle) = \twopartdef{x}{\mbox{STP}(\phi_e, t) = 1}{k}{\mbox{si no.}} \]
\noindent $\func{f}{\naturals}{\naturals}$ es computable (total) y además $f(\naturals) = A$, ya que si $x \in A$ entonces $\phi_e(x) \downarrow$ en algún paso $t$ de la ejecución y por lo tanto $f(\langle x, t \rangle) = x$.

\noindent Queda ver que $f$ es p.r., pero podemos expresar a $f$ como la composición de funciones p.r.
\[ f(\langle x, t \rangle) = x \, \mbox{STP}(\phi_e, t) \, + \, k \, \alpha(\mbox{STP}(\phi_e, t)) \]
\noindent ya que la suma, STP, $\alpha$ y los observadores de tuplas son p.r. Recordamos que $\alpha$ es:
	
	\[ \alpha(x) = \twopartdef{1}{x = 0.}{0}{\mbox{si } x \neq 0.}\] 
	
\end{solucion}

\begin{ej}{1}
    La clase de funciones computables es una clase $\bm{PRC}$
\end{ej}

\begin{solucion}
Tenemos que mostrar que si una funcion es computable entonces podemos llegar por medio de las iniciales o por composición o recursion primitiva.
En principio tenemos que las iniciales estan  y es facil de ver con programas en $S$:

\begin{itemize}
\renewcommand\labelitemi{$\sim$}
\item $n(x) = 0$ es el programa vacio,
\item $s(x) = x + 1$ es el programa   $Y \leftarrow X + 1  $
\item $u_i^n(x_1, \dots, x_n) = x_i$ es el programa $Y \leftarrow X_i  $
\item Composición: $f$, $g_1, \dots, g_k$ parciales computables entonces podemos expresar la composición como el siguiente programa 
\begin{align*}
	 & Z_1 \leftarrow g_1(X) \\
	 & Z_2 \leftarrow g_2(X) \\
	 & \vdots \\
	 & Z_k \leftarrow g_k(X) \\
	 & Y \leftarrow f(Z_1, \cdots, Z_k)
\end{align*}
\item Recursión primitiva: $f$, $g_1, \dots, g_k$ parciales computables entonces podemos expresar la recursión primitiva con el siguiente programa 
\begin{align*}
	    	  & Y \leftarrow f(X_x, \dots, X_n) \\
	[C] \quad & \mbox{IF } X = X_{n+1} \mbox{ GOTO } E \\
			  & Y \leftarrow g(Y, X_1, \dots X_n, T) \\
			  & T \leftarrow T + 1 \\
			  & \mbox{ GOTO } C 
\end{align*}
\end{itemize}

\noindent Si teniamos funciones parciales computables estos programas muestran que tanto la composición como la recursión primitiva nos da una funcion parcial computable y además como las funciones iniciales son computables tenemos que las funciones computables son una clase $\PRC$
\end{solucion}

\begin{ej}{1}
    Probar que si $\ba$ y $\bm{B}$ son $c.e.$ entonces también lo son $\ba \cup
    \bm{B}$ y $\ba \cap \bm{B}$.
\end{ej}

\begin{solucion}
\noindent Como $\ba$ y $\bb$ son c.e. existe $f$ y $g$ parcial computables tal que 
	\[ \ba = \dom(f) = \set{x}{f(x) \downarrow} \mbox{ \quad y \quad } 
	   \bb = \dom(g) = \set{x}{g(x) \downarrow}\] 
Definimos entonces para la unión
	\[ h(x) = \twopartdef{\downarrow}{\min\limits_t(\STP(x, f, t) \lor \STP(x, g, t))}
									 {\uparrow}{\mbox{si no.}} \]
Y con la misma idea la intersección
	\[ h(x) = \twopartdef{\downarrow}{f(x) \land g(x)}
					     {\uparrow}{\mbox{si no.}} \]
Estas funciones son parcial computables y computan la unión y la intersección.
\end{solucion}

\begin{ej}{1}
    Exhibir una función computable que no sea $p.r.$ y demostrarlo.
\end{ej}

\begin{solucion}
Podemos listar todas las funciones p.r.: $\phi_1, \phi_2, \dots$, entonces definamos la funcion \[ f(x) = \phi_{x}(x) \] esta función devuelve la e-esima funcion p.r. evaluado en su número de programa. Supongamos que $f$ es p.r., entonces también es p.r. (por composición) \[ g(x) = f(x) + 1 \] Pero esto implica que $g$ estaba en la lista de funciones p.r., osea hay un $k$ tal que $\phi_k = g$ de lo cual obtenemos que \[ \phi_k(x) = g(x) = f(x) + 1 = \phi_x(x) + 1 \] Si tomamos $x = k$ llegamos a una contradicción que vino de suponer que $f$ era p.r. 
\end{solucion}

\begin{ej}{1}
    Enunciar y demostrar el Teorema de punto fijo.
\end{ej}

\begin{solucion}
\begin{teo}{}
Si $\func{f}{\naturals}{\naturals}$ parcial computable entonces existe $e$ tal que $\phi_e = \phi_{f(e)}$
\end{teo}

\begin{dem}
Consideremos la función \[ g(x, y) = \phi_{f(x)}(y) \] Si aplicamos el teorema de la recursión tenemos un $e$ tal que \[ g(e, y) = \phi_{e}(y) \] Pero entonces por la definición de $g$ es equivalente a que
	\[ \phi_{e}(y) = \phi_{f(e)}(y) \]
\end{dem}


\end{solucion}


\section*{Lógica Proposicional}

\begin{ej}{5}
    Usando el teorema de correctitud de la lógica proposicional, probar que
    $\bm{\Gamma}$ es un conjunto de fórmulas satisfacible entonces es
    consistente.
\end{ej}

\begin{solucion}
Queremos ver que si $\bgamma$ es satisfacible (existe $v$ tal que $v \models \bgamma$ entonces es consistente. 
\begin{dem} 
Supongamos que es inconsistente, entonces existe $\psi$ tal que 
\[ \bgamma \vdash \psi  \mbox { \, y \, } \bgamma \vdash \lnot\psi \] 
Pero por correctitud esto significa que
\[ \bgamma \models \psi \mbox { \, y \, } \bgamma \models \lnot\psi \] 
Entonces $v \models \psi$ y $v \models \lnot\psi$ lo cual es una contradicción, vino de suponer que $\bgamma$ no era consistente.
\end{dem}
\end{solucion}

\begin{ej}{4}
    \begin{enumerate}
    \item Definir consistente, maximal consistente.
    \item Enunciar y demostrar el teorema de Lindenbaum para lógica proposicional.
    \end{enumerate}
\end{ej}

\begin{solucion}
\textbf{1.} Un conjunto $\bm{\Gamma}$ es maximal consistente si: es consistente y para toda formula $\varphi$ tenemos que
\begin{enumerate}
	\item $\varphi \in \bm\Gamma$ ó
	\item Existe un $\psi$ tal que $\bm{\Gamma} \cup \{\varphi\} \vdash \psi$  y $\bm{\Gamma} \cup \{\varphi\} \vdash \lnot\psi$ entonces
\end{enumerate}

\begin{thm}[Lema de Lindenbaum] Si $\bm{\Gamma} \subseteq \FORM$ entonces existe un $\bm{\Gamma'}$ m.c. tal que $\bm{\Gamma} \subseteq \bm{\Gamma'}$
\end{thm}

\noindent Queremos expandir $\bgamma$ de manera que sea maximal y no deje de ser consistente. Consideremos entonces una numeración de todas las fórmulas $\varphi_1, \varphi_2, \cdots$ y definimos los siguientes conjuntos

\begin{itemize}
\renewcommand\labelitemi{$\sim$}
\item $\bgamma_0 = \bgamma$
\item $\bgamma_{n+1} = 
	\twopartdef{\bgamma_n \cup \{\varphi_{n + 1}\}}{\bgamma_n \cup \{\varphi_{n + 1}\} \mbox{es consistente.}}
			   {\bgamma_n}{\mbox{si no.}}$
\item $\bgamma' = \bigcup\limits_{k \geq 0} \bgamma_k$
\end{itemize}

\noindent Entonces afirmamos que $\bgamma'$ es m.c.

\noindent \textbf{Consistencia:} Supongamos que no lo es, entonces hay una formula $\psi$ tal que 
\[\bgamma \vdash \psi \mbox{ \, y \, } \bgamma \vdash \lnot \psi\]

\noindent Pero esto significa que en sus derivaciones hay un número finito de fórmulas
\[\{\gamma_1, \gamma_2, \cdots, \gamma_n\} \subseteq \bgamma'\]
\noindent Por lo tanto hay un $k$ lo suficientemente grande tal que 
\[\{\gamma_1, \gamma_2, \cdots, \gamma_n\} \subseteq \bgamma_k\]
pero entonces $\bgamma_k$ tampoco es consistente, lo cual contradice nuestra construcción.\\
 
\noindent\textbf{Maximal:} La maximalidad significa que dada una fórmula $\varphi$ o bien $\varphi \in \bgamma'$ ó si se la agregamos a $\bgamma'$ se hace inconsistente. \\ 
\noindent Entonces supongamos que $\varphi \not\in \bgamma'$. Tenemos que $\varphi$ debe aparecer en la numeración que hicimos sobre todas las fórmulas, por ejemplo en $\varphi_n$, pero entonces deberíamos haberla agregado en $\bgamma_n$, de lo cual tenemos que $\varphi_n \not\in \bgamma_n$ por lo tanto $\bgamma_{n - 1} \cup \varphi_n$ es inconsistente, de lo cual tenemos que $\bgamma' \cup \varphi_n$ es inconsistente.

\end{solucion}


\begin{ej}{3}
    Demostrar que si $\bm{\Gamma}$ es un conjunto de fórmulas de logica proposicional
    tal que $\bm{\Gamma \subseteq FORM}$ es consistente entonces $\bm{\Gamma}$ es
    satisfacible.
\end{ej}

\begin{solucion}
Supongamos que es consistente, queremos encontrar una valuación que satisface a $\bgamma$, la idea va a venir por construir el conjunto m.c. que extiende a $\bgamma$, definir una valuación que hace verdad al mismo y ver que esa valuación también satisface a $\bgamma$ \\

\begin{dem}
Sea $\bgamma \subseteq \FORM$ consistente, entonces por el teorema de Lindenbaum podemos extenderlo a un
 conjunto maximal consistente llamado $\bgamma'$. Definimos entonces la valuacion
  \[ v(p) = 1  \Leftrightarrow p \in \bgamma' \]
Veamos que con esta valuación nos alcanza, vamos entonces a demostrar lo siguiente, $v \models \varphi$ sii $\varphi \in \bgamma'$. Lo haremos por inducción en la complejidad de (cantidad de $\lnot$ y $\rightarrow$ que aparecen) $\varphi$.

\noindent \textbf{Caso base:} $\varphi = p$, es trivial por la definición de $v$.

\noindent Recordamos nuestra hipótesis inductiva: Si $\psi$ es una formula con complejidad $\leq n$ entonces \[ v \models \psi \mbox{ \, si y solo si \,} \psi \in \bgamma' \]
\textbf{Paso inductivo:} Supongamos que tenemos una fórmula $\varphi$ con longitud $n + 1$ entonces o bien
$\varphi = \lnot \psi$ ó $\varphi = \piz$.\\

\noindent \textbf{Caso $\lnot \psi$:} \\ 
\noindent $\Rightarrow)$ Si $v \models \varphi$ entonces $v \models \lnot\psi$ de lo cual $v \not\models \psi$, pero entonces la complejidad de $\psi$ es menor que $n + 1$, aplicando la \textbf{HI} tenemos que $\psi \not\in \bgamma'$ y como este es m.c. debe contener a $\lnot\psi$ osea que contiene a $\varphi$. \\
\noindent $\Leftarrow)$ Si $\lnot\psi \in \bgamma'$ entonces como $\bgamma'$ m.c., $\psi \not\in \bgamma'$, aplicando la \textbf{HI} de vuelta tenemos que $v \not\models \psi$ osea que $v \models \lnot \psi$ y esto es lo mismo que $v \models \varphi$. \\

\noindent \textbf{Caso $\piz$:} \\ 
\noindent $\Rightarrow)$ Si $v \models \piz$, entonces $v \not\models \psi$ ó $v \models \zeta$, supongamos que pasa lo primero, entonces 
\[v \not\models \psi \overset{HI}{\implies} \psi \not\in\bgamma' \implies \lnot\psi \in \bgamma' \implies \bgamma' \vdash \lnot\psi \]
Usando el siguiente teorema $\vdash \lnot\psi \rightarrow (\piz)$ y por $MP$ tenemos $\bgamma \vdash \piz$ entonces $\piz \in \bgamma'$. \\
Por otro lado tenemos que 
	\[ v \models \zeta \overset{HI}{\implies} \zeta \in \bgamma' \implies \bgamma' \vdash \zeta \]
Usando SP1 sabemos que $\vdash \zeta \rightarrow (\piz)$ y aplicando $MP$ entre esto y lo anterior tenemos que $\bgamma' \vdash \piz$, osea que $\piz \in \bgamma'$.\\

\noindent $\Leftarrow)$ Si $v \not\models \piz$ esto pasa si $v \models \psi$ y $v \not\models \zeta$ aplicando la \textbf{HI} tenemos que $\psi \in \bgamma'$ y $\lnot\zeta \in \bgamma'$, osea que 
\[\bgamma \vdash \lnot\zeta \mbox{\, y \,} \bgamma \vdash \psi\] 
Usando el siguiente teorema 
\[ \vdash \psi \rightarrow (\lnot \zeta \rightarrow \lnot(\piz))\]
y por $MP$ tenemos que
\[ \lnot\zeta \rightarrow \lnot(\piz)\]
$MP$ una vez más y obtenemos que $\bgamma' \vdash \lnot(\piz)$. Osea que $\lnot(\piz) \in \bgamma'$, entonces $\piz \not\in \bgamma'$.
\end{dem}
\end{solucion}


\begin{ej}{2}
    Enunciar y demostrar el teorema de la deducción.
\end{ej}

\begin{solucion}
\begin{teo}{} Si $\bgamma \cup \pset{\varphi} \vdash \psi$ entonces $\bgamma \vdash 
				  \varphi \rightarrow \psi $ \end{teo} \\
				  
\begin{dem}
	Haremos inducción en la longitud de la demostración de $\bgamma \cup \pset{\varphi} \vdash \psi$.\\
	
\noindent\textbf{$n = 1$}: tenemos que 3 casos, $\psi \in \bgamma$, $\psi$ es un axioma ó $\psi = p$.

Si $\psi \in \bgamma$ entonces

\begin{align}
& \psi \\
& \psi \rightarrow (\varphi \rightarrow \psi) \mbox{ Por SP1 } \\
& \varphi \rightarrow \psi
\end{align}

\noindent Si $\psi$ es un axioma entonces, también

\begin{align}
& \psi \\
& \psi \rightarrow (\varphi \rightarrow \psi) \mbox{ Por SP1 } \\
& \varphi \rightarrow \psi
\end{align}

\noindent Si $\psi = p$ vale el teorema $\vdash p \rightarrow p$.\\

\noindent \textbf{Paso inductivo:} Supongamos que lo probamos hasta $n + 1$ en la longitud de la demostracion
entonces ahora tenemos 4 casos, 3 son los mismos que ya vimos y se prueban trivialmente, y ahora se agrega un cuarto que es que vale por $MP$. \\

\noindent Si $\psi$ se demuestra por $MP$ a partir de $\varphi_k, \varphi_i = \varphi_k \rightarrow \psi$  \\
Entonces por \textbf{HI} tenemos que
 \[ \bgamma \cup \pset{\varphi} \vdash \varphi_k \implies \bgamma \vdash \varphi \rightarrow \varphi_k\]
 \[ \bgamma \cup \pset{\varphi} \vdash \varphi_i \implies \bgamma \vdash \varphi \rightarrow \varphi_i \implies \bgamma \vdash \varphi \rightarrow (\varphi_k \rightarrow \psi) \]
Usando el teorema 
\[\vdash (\varphi \rightarrow (\varphi_k \rightarrow \psi)) \rightarrow ((\varphi \rightarrow \varphi_k) \rightarrow (\fis))\]
 Aplicando $MP$ dos veces obtenemos $\bgamma \vdash (\varphi \rightarrow \psi)$
\end{dem}
\end{solucion}

\begin{ej}{1}
    $\bgamma \union \{ \lnot\varphi \}$ es inconsistente $\sii \bgamma \vdash \varphi$. (En lógica proposicional) 
\end{ej}

\begin{solucion}

$\Leftarrow$) Como $\bgamma \vdash \varphi$ si agregamos $\lnot\varphi$ tenemos que
	 \[\grab{\bgamma \cup \pset{\lnot\varphi} \vdash \varphi}
	 	    {\bgamma \cup \pset{\lnot\varphi} \vdash \lnot\varphi}
	 	    {\bgamma\cup\pset{\lnot\varphi} \mbox{ es inconsistente.}}\]
$\Rightarrow$) Como $\bgamma \cup \pset{\lnot\varphi}$ es inconsistente entonces existe un $\psi$ tal que
\[ \bgamma \cup \pset{\lnot\varphi} \vdash \psi \mbox{\, y \,} \bgamma \cup \pset{\lnot\varphi} \vdash \lnot \psi \]
Por el teorema de la deducción tenemos que esto es lo mismo que
\[ \bgamma \vdash \lnot\varphi \rightarrow \psi \mbox{\, y \,} 
   \bgamma \vdash \lnot\varphi \rightarrow \lnot \psi \]
Usando el teorema 
\[ \vdash (\lnot\varphi \rightarrow \psi) \rightarrow ((\lnot\varphi \rightarrow \psi) \rightarrow \varphi) \]
Si aplicamos MP con lo que obtuvimos del teorema de la deducción obtenemos
\[ \vdash ((\lnot\varphi \rightarrow \psi) \rightarrow \varphi) \]
Y si aplicamos MP una vez más
\[ \varphi \]
\end{solucion}

\begin{ej}{2}
    Definir conjunto maximal consistence. \\
    Probar que si $\bgamma$ es maximal consistente entonces:
    \begin{enumerate}
        \item $\varphi \in \bgamma$ ó (excluyente) $\lnot\varphi \in \bgamma$.
        \item $\varphi \in \bgamma \sii \bgamma \vdash \varphi$ (consecuencia sintátctica).
    \end{enumerate}
\end{ej}

\begin{solucion}
\begin{defi}
$\bgamma \subseteq \FORM$ es m.c. si es consistente y para toda fórmula $\varphi$ o bien $\varphi \in \bgamma$ o existe un $\psi$ tal que 
\[\bgamma \cup \pset{\varphi} \vdash \psi \mbox{\, y\, }  \bgamma \cup \pset{\varphi} \vdash \lnot\psi\]
\end{defi}


\noindent\textbf{1.} Supongamos que ni $\varphi$ ni $\lnot\varphi$ estan en $\bgamma$, entonces tenemos que como es m.c.
	\[ \bgamma \cup \pset{\varphi} \mbox{ es inconsistente } \implies \bgamma \vdash \lnot\varphi \]
	\[ \bgamma \cup \pset{\lnot\varphi} \mbox{ es inconsistente } \implies \bgamma \vdash \varphi \]
Osea que $\bgamma$ es inconsistente para $\varphi$. Entonces alguno de los 2 tiene que pertenecer. \\

\noindent\textbf{2.}
\begin{dem}
$\Rightarrow)$ $\bgamma \vdash \varphi \implies \bgamma \cup \pset{\lnot\varphi}$ es inconsistente entonces $\varphi \in \bgamma$ (Usando el ejercicio anterior). \\
\noindent $\Leftarrow)$ Si $\varphi \in \bgamma $ entonces $\bgamma \vdash \varphi$ ya que $\varphi$ es axioma o está por consecuencia inmediata. 
\end{dem}

\end{solucion}




\section*{Lógica de primer orden}

\begin{ej}{8}
    Sea $\bm{\mathcal{L}} = \{0, S, <, +, *\}$ con igualdad y sea
    $\bm{\mathcal{N}} = \{\mathbb{N},
    0, S, <, +, *\}$ la $\bm{\mathcal{L}}$-estructura de primer orden con la
    interpretación usual. \\ Mostrar que existe un modelo de los naturales en donde
    valen todas las verdades de N, pero donde existe un elemento inalcanzable
    (desde el 0, usando la función sucesor).
\end{ej}

\noindent Variaciones del ejercicio anterior:

\begin{ej}{1}
    Demostrar que existen modelos de primer orden no estandard para la
    aritmética.
\end{ej}

\begin{ej}{1}
    Mostrar que existe un modelo no estandard de la aritmética.
\end{ej}

\begin{ej}{1}
    Demostrar que existen modelos no estándard de la aritmética en los que hay
    un elemento inalcanzable.
\end{ej}

\begin{solucion}
\begin{dem}
	Lo vamos a demostrar para un elemento inalcanzable aunque se podría hacer para la no igualdad (Ejemplo en el libro de metalógica). Sea
	 \[\TeoN = \set{\varphi \in \FORM}{ \varphi \mbox{ es sentencia y } \ARITH \models \varphi}\]
\noindent Entonces expandamos el lenguaje con una nueva constante $c$ y los siguientes axiomas
 \[ \bm{\Gamma} = \{ 0 < c, S(0) < c, S(S(0)) < c, \dots \} \]

\noindent Cada subconjunto finito de $\bm{\Gamma} \cup \TeoN$ tiene un modelo, ya que tenemos finitos axiomas de $\Gamma$ y por lo tanto existe un $c \in \naturals$ que cumple la condición.\\
\noindent Además por compacidad $\TeoN \cup \bm{\Gamma}$ tiene modelo. \\
\noindent Y por Löwenheim-Skolem tiene un modelo contable.

	\[ \MODEL = \langle M, 0^{\MODEL}, S^{\MODEL}, <^{\MODEL}, +^{\MODEL}, *^{\MODEL}, c^{\MODEL} \rangle \]

\noindent entonces tomemos $\MODEL'$ la restricción de $\MODEL$ al lenguaje original $\mathcal{L}$ de lo cual vemos que $\ARITH \models \varphi \iff \MODEL' $

\[ \ARITH \models \varphi \implies \varphi \in \TeoN \implies \MODEL \models \varphi \implies \MODEL' \models \varphi \]
\[ \ARITH \models \lnot\varphi \implies \lnot\varphi \in \TeoN \implies \MODEL \models \lnot\varphi \implies \MODEL' \models \lnot\varphi \]

\noindent Pero $\ARITH$ y $\MODEL'$ no son isomorfos ya que $c^{\MODEL}$ es un elemento inalcanzable en $\MODEL'$ es decir obtuvimos un modelo no estandar de la aritmética.

\end{dem}
\end{solucion}

\begin{solucion}

\textbf{Solución alternativa} (para alguno de los anteriores):
como un modelo estandar de la aritmética tiene un modelo infinito, por Löwenheim-Skolem tiene modelos de cardinalidad arbitraria ($\geq \aleph_0$) entonces un modelo no numerable de los naturales no puede ser estandar.

\end{solucion}

\begin{ej}{4}
    Dado $\bm{L}$ un lenguaje de primer oden con igualdad. Decidir si las
    siguientes afirmaciones son verdaderas o falsas.

    \begin{enumerate}
        \item Existe un conjunto $\bgamma$ tal que $\bgamma \models \ba$ si  y
            solo si $\ba$ tiene universo infinito.
        \item Existe un conjunto $\bgamma$ tal que $\bgamma \models \ba$ si y
            solo si $\ba$ tiene universo finito.
        \item El conjunto $\bgamma$ del ítem 1 necesariamente es infinito.
    	\item ?` $\exists\varphi$ tal que $\bm{M} \models \varphi$ $\sii$ modelo que la satisface es finito? ?`Por qué?
    	\item  Verdadero o Falso: $\bm{M} \models \varphi \sii \bm{M}$ es infinito.
    \end{enumerate}

\end{ej}


\begin{ej}{2}
    Probar que si una teoría de primer orden con igualdad tiene modelos
    arbitrariamente grandes, entonces tiene modelos infinitos.
\end{ej}

\begin{ej}{1}
   Enumerar (y explicar brevemente) los pasos de la demostración de completitud
    en primer orden y mostrar el modelo canónico utilizado en la Demostración.
\end{ej}

\begin{ej}{1}
    Sea $\mathcal{L} = \{c, f\}$ un lenguaje de primer orden con igualdad donde
    $c$ es un símbolo de constante y $f$ un símbolo de función unaria.
    \begin{enumerate}
        \item Definir $\varphi, \psi \in \FORM(\mathcal{L})$ tal que:
            \begin{enumerate}
                \item $\varphi$ sea verdadera sii $c$ no pertenece al rango de
                    $f$
                \item $\psi$ sea verdadera sii $f$ es inyectiva
            \end{enumerate}
        \item Para $\theta = (\varphi \land \psi)$, probar que si $\ba \models
            \theta$ entonces $\ba$ es infinito.
        \item Definir $\theta' \in \FORM(\mathcal{L})$ tal que si $\ba$ es
            infinito entonces $\ba \models \theta'$.
        \item ?`Existe $\theta'' \in \FORM(\mathcal{L})$ tal que si $\ba$ es
            infinito si y solo si $\ba \models \theta''$? justificar.
    \end{enumerate}
\end{ej}

\end{document}
